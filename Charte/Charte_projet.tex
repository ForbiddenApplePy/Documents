\documentclass[14pt,a4paper]{extarticle}
\author{Valentin MICHEL}
\date{4 avril 2020}
\usepackage{graphicx}
\usepackage{hyperref}
\usepackage[document]{ragged2e}
\usepackage[a4paper, tmargin=3cm]{geometry}
%\pagenumbering{gobble} 
%\makeatletter
%\renewcommand{\@seccntformat}[1]{}
%\makeatother

\title{Charte de Projet \\ Projet ForbiddenApplePy}
\begin{document}
\maketitle{}
%\center{\includegraphics[]{../../supertest.jpg}}
\justify
\break
\tableofcontents
\break
\section{Cadrage du projet}
\subsection{Délais de Réalisations}
Le projet s'étend sur la quasi totalité d'un semestre IN'TECH, soit de la première semaine d'avril 2020 à fin juin 2020.

\subsection{Directeur de Projet}
Le directeur de projet est David Huet, professeur référent de la filière Système et Réseau.

\break
\subsection{Chef de Projet et Équipe de réalisation}
L'équipe NetScratch est constituée de 2 membres :
\begin{itemize}
    \item{\textbf{Olivier Lespagnon} - \textit{Chef de Projet}}
    \item{Valentin Michel}
\end{itemize}

\break
\section{Gestion du Reporting}
Le reporting du projet sera effectué toutes les semaines, lors du dernier créneau PI.
Les éventuelles informations confidentielles seront envoyées séparément
\section{Calendrier}
Le projet est découpé en 4 itérations, dont les dates sont encore à fixer :
\begin{itemize}
    \item{Itération 1 : Avant Projet - TO DEFINE}
    \item{Itération 2 : Installation et configuration}
    \item{Itération 3 : Automatisation}
    \item{Itération 4 : Recette}
\end{itemize}
Le cadrage de projet et la tenue de calendrier se fera avec l'outil \href{https://trello.com/b/XQZtrqGi}{Trello} \\
Ce dernier sera mis à jour au minimum une fois par semaine, le vendredi.
\section{Gestion de la Documentation}
La documention sera rédigée en \Latex et sera hébergé sur github

\break
